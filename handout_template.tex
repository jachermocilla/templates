\documentclass[a4paper, 11pt,oneside]{article}
\usepackage[
  top=1.5cm,
  bottom=1cm,
  left=2cm,
  right=1.5cm,
  headheight=25.22153pt, % as per the warning by fancyhdr
  includehead,includefoot,
  heightrounded, % to avoid spurious underfull messages
]{geometry} 

\usepackage[T1]{fontenc}
\usepackage{microtype}
\usepackage{fancyhdr}
\usepackage{fancyvrb}
\usepackage{lipsum}
\usepackage{url}
\usepackage{listings}

\pagestyle{fancy}
\fancyhf{} % clear all fields

\pagestyle{fancy}
\lhead{CMSC 125: Operating Systems \\ Second Semester 2017-2018}
\rhead{Institute of Computer Science \\ University of the Philippines Los Banos}
\rfoot{JACHermocilla (jchermocilla@up.edu.ph)}
\renewcommand{\headrulewidth}{0.4pt}
\renewcommand{\footrulewidth}{0.4pt}

\begin{document}

\begin{center}
	{\LARGE \textbf{Homework 1: Building ICS-OS}}
\end{center}

\section*{Objectives}
   At the end of this activity, you should be able to:
   \begin{enumerate}
       \item build the ICS-OS kernel and disk image;
       \item run ICS-OS in QEMU and
       \item run two ICS-OS commands.
   \end{enumerate}   

\section{Introduction}
ICS-OS\footnote{https://github.com/srg-ics-uplb/ics-os/} is an instructional (not for production) operating system that can be used for understanding different operating system concepts. An 
operating system is no different from other software in that it is written in a programming language,  
such as C. Later in the course, you will be modifying portions of the source code of ICS-OS to 
apply and observe various operating system concepts. The tasks  in this homework are from the ICS-OS Kernel Developer's Guide\footnote{https://github.com/srg-ics-uplb/ics-os/wiki/Kernel-Developer's-Guide}.

%\section{Prerequisites}

\section{Deliverables}
Perform the tasks below and capture screen shots. Submit a PDF file 
containing the screen shots. 


\section{Tasks}

\subsection*{Task 1: Install dependencies}
\begin{lstlisting}[language=bash,frame=single]
$sudo apt-get update
$sudo apt-get install build-essential nasm qemu-kvm tcc git gcc-multilib
\end{lstlisting}

\subsection*{Task 2: Clone the repository}
\begin{lstlisting}[language=bash,frame=single] 
$git clone https://github.com/srg-ics-uplb/ics-os.git
$cd ics-os/ics-os
\end{lstlisting}

\subsection*{Task 3: Build}
Building the source code for the kernel and the distribution disk is accomplished using make. Make sure you perform steps 2-4 every time you make changes in the source code.
\begin{lstlisting}[language=bash,frame=single] 
$make clean
$make
$make floppy 
\end{lstlisting}

\subsection*{Task 4: Run}
\begin{lstlisting}[language=bash,frame=single] 
$make run-floppy 
\end{lstlisting}

\subsection*{Task 4: Run ICS-OS commands}
Once the ics-os command prompt appears, type help. 
Examine the list of commands and run two commands.

%\begin{thebibliography}{9}
%\end{thebibliography}



\end{document}